\documentclass[a4paper,titlepage]{article}
\usepackage{hyperref}
\usepackage[T1]{fontenc}
\usepackage[utf8]{inputenc}
\usepackage[italian]{babel}
\usepackage{graphicx}
\usepackage{epstopdf}
\usepackage{fancyhdr}
\usepackage[margin=3.5cm]{geometry}
%\usepackage[margin=3cm, bottom=3.0cm, top=3.0cm, noheadfoot]{geometry}
\usepackage{lastpage}
\usepackage{wrapfig}
\usepackage{float} % posizionamento immagini %
\usepackage{listings}
\usepackage{makeidx}
\usepackage{subfig}
\makeindex

\renewcommand{\headrule}{\hbox to\headwidth{\dotfill} \vskip 0.25cm}
\lstset{language=C++, basicstyle=\small\sffamily, numbers=left, numberstyle=\tiny,
frame=tb, columns=fullflexible, showstringspaces=false}

\setlength{\parindent}{0in}
\newcommand{\sectrule}{\rule[0.2cm]{15cm}{0.05cm}\\[0.001cm]}
\newcommand{\subsectrule}{\rule{10cm}{0.025cm}\\}
\newcommand{\mobius}{M\"{o}bius }
\author{Alessandro Rosetti}
\title{Tesi}

\usepackage{fancyhdr}
\pagestyle{fancy}
%\renewcommand{\chaptermark}[1]{\markboth{\thechapter.\ #1}{}}
%\renewcommand{\sectionmark}[1]{\markright{\thesection.\ #1}} 
\fancyhf{}
\fancyhead[RO,LE]{\bfseries\thepage}
\fancyhead[LO]{\bfseries\rightmark}
\fancyhead[RE]{\bfseries\leftmark}
\renewcommand{\headrulewidth}{0.4pt}
\renewcommand{\footrulewidth}{0pt}
 
\fancypagestyle{plain}{
        \fancyhf{}
        \renewcommand{\headrulewidth}{0pt}
        \renewcommand{\footrulewidth}{0pt}
}
 
% rimuove l'header dalle pagine bianche:
\def\cleardoublepage{\clearpage\if@twoside \ifodd\c@page\else
  \hbox{}
  \thispagestyle{empty}
  \newpage
  \if@twocolumn\hbox{}\newpage\fi\fi\fi}
 
\usepackage{calc}
\addtolength{\textwidth}{-0.5 in}
\addtolength{\hoffset}{0.25 in}
\setlength{\headwidth}{\paperwidth}
\addtolength{\headwidth}{-1 in}
\addtolength{\headwidth}{-\hoffset}
\addtolength{\headwidth}{-\evensidemargin}
\addtolength{\headwidth}{-\evensidemargin}

%******************************************************%
\begin{document}
\begin{titlepage}
\begin{center}
\huge{\textbf{UNIVERSIT\`A DEGLI STUDI DI PISA}}\\
\LARGE{Facoltà di Ingegneria}
\begin{figure}[H]
  \centering \includegraphics[scale=0.35]{img/unipi.jpg}
\end{figure}
\vspace{2.5cm}
\Huge{Progetto di Sicurezza}\\[2.5cm]

\large{\textbf{Autori}}\\
\underline{\Large{Alessandro Rosetti}}\\
\underline{\Large{Daniele Lazzarini}}\\
\underline{\Large{Alessandro Furlanetto}}\\


\vfill
\large{Anno accademico 2011-2012}\\[2.0cm]
\end{center}
\end{titlepage}

%******************************************************%
\begin{center}
\thispagestyle{empty}
\newpage
\vfill
Documento sviluppato con \LaTeX\\
\today
\end{center}

\newpage

\tableofcontents \thispagestyle{fancy}
\newpage
\listoffigures
\newpage

\section*{Introduzione} \addcontentsline{toc}{section}{Introduzione} \thispagestyle{plain}
Il progetto illustrato in questo documento si chiama \textbf{pschat} e rappresenta un sistema client-server di chat ispirato al modello \textbf{IRC} (Internet Relay Chat) che integra caratteristiche di sicurezza.\\
\begin{figure}[H]
  \centering \includegraphics[scale=1.0]{../data/psc_orig.png}
\end{figure}

\section{Protocollo}
L'autenticazione è stata eseguita sfruttando RSA.
\subsection{Server}
Il server ha una chiave privata e una pubblica (nota a priori per ogni client), è dotato di un database che contiene i nomi degli utenti, gli hash delle loro password e altre informazioni.\\
Il server conosce tutte le chiavi pubbliche degli utenti registrati al sistema.

\subsection{Client}
Il client ha una chiave privata e pubblica e conosce la chiave pubblica del server.\\
\newpage


\end{document}
