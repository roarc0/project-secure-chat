\documentclass[a4paper,titlepage]{article}
\usepackage{hyperref}
\usepackage[T1]{fontenc}
\usepackage[utf8]{inputenc}
\usepackage[italian]{babel}
\usepackage{graphicx}
\usepackage{epstopdf}
\usepackage{fancyhdr}
\usepackage[margin=3.5cm]{geometry}
%\usepackage[margin=3cm, bottom=3.0cm, top=3.0cm, noheadfoot]{geometry}
\usepackage{lastpage}
\usepackage{wrapfig}
\usepackage{float} % posizionamento immagini %
\usepackage{listings}
\usepackage{makeidx}
\usepackage{subfig}
\usepackage{amssymb} % symbols
\makeindex

\renewcommand{\headrule}{\hbox to\headwidth{\dotfill} \vskip 0.25cm}
\lstset{language=C++, basicstyle=\small\sffamily, numbers=left, numberstyle=\tiny,
frame=tb, columns=fullflexible, showstringspaces=false}

\setlength{\parindent}{0in}
\newcommand{\sectrule}{\rule[0.2cm]{15cm}{0.05cm}\\[0.001cm]}
\newcommand{\subsectrule}{\rule{10cm}{0.025cm}\\}
\newcommand{\mobius}{M\"{o}bius }
\author{Alessandro Rosetti}
\title{Tesi}

\usepackage{fancyhdr}
\pagestyle{fancy}
%\renewcommand{\chaptermark}[1]{\markboth{\thechapter.\ #1}{}}
%\renewcommand{\sectionmark}[1]{\markright{\thesection.\ #1}} 
\fancyhf{}
\fancyhead[RO,LE]{\bfseries\thepage}
\fancyhead[LO]{\bfseries\rightmark}
\fancyhead[RE]{\bfseries\leftmark}
\renewcommand{\headrulewidth}{0.4pt}
\renewcommand{\footrulewidth}{0pt}
 
\fancypagestyle{plain}{
        \fancyhf{}
        \renewcommand{\headrulewidth}{0pt}
        \renewcommand{\footrulewidth}{0pt}
}
 
% rimuove l'header dalle pagine bianche:
\def\cleardoublepage{\clearpage\if@twoside \ifodd\c@page\else
  \hbox{}
  \thispagestyle{empty}
  \newpage
  \if@twocolumn\hbox{}\newpage\fi\fi\fi}
 
\usepackage{calc}
\addtolength{\textwidth}{-0.5 in}
\addtolength{\hoffset}{0.25 in}
\setlength{\headwidth}{\paperwidth}
\addtolength{\headwidth}{-1 in}
\addtolength{\headwidth}{-\hoffset}
\addtolength{\headwidth}{-\evensidemargin}
\addtolength{\headwidth}{-\evensidemargin}

%******************************************************%
\begin{document}
\begin{titlepage}
\begin{center}
\huge{\textbf{UNIVERSIT\`A DEGLI STUDI DI PISA}}\\
\LARGE{Facoltà di Ingegneria}
\begin{figure}[H]
  \centering \includegraphics[scale=0.35]{img/unipi.jpg}
\end{figure}
\vspace{2.5cm}
\Huge{Progetto di Sicurezza}\\[2.5cm]

\large{\textbf{Autori}}\\
\underline{\Large{Alessandro Rosetti}}\\
\underline{\Large{Daniele Lazzarini}}\\

\vfill
\large{Anno accademico 2011-2012}\\[2.0cm]
\end{center}
\end{titlepage}

%******************************************************%
\begin{center}
\thispagestyle{empty}
\newpage
\vfill
Documento sviluppato con \LaTeX\\
\today
\end{center}

\newpage

\tableofcontents \thispagestyle{fancy}
\newpage
%\listoffigures
%\newpage

\section{Introduzione} \addcontentsline{toc}{section}{Introduzione} \thispagestyle{plain}
Il progetto illustrato in questo documento si chiama \textbf{pschat} e rappresenta un sistema client-server di chat ispirato al modello \textbf{IRC} (Internet Relay Chat) che integra caratteristiche di sicurezza.\\
\begin{figure}[H]
  \centering \includegraphics[scale=1.0]{../data/psc_orig.png}
\end{figure}

\subsection{Formato del pacchetto}

Pacchetto \textbf{RSA} OEAP:
\begin{verbatim}
[LEN]  RSA-ENCRYPTED{ [OPCODE][LEN][PAYLOAD] }
  2            2      2    0-252
\end{verbatim}
$Size = 2 + 256= 258$ Bytes.\\

Pacchetto \textbf{AES-128/256}:
\begin{verbatim}
[LEN]  { [IV] } AES-ENCRYPTED { [OPCODE][LEN][SEQ][PAYLOAD] }
  2       16                       2      2    4   0-65000
\end{verbatim}

$Size_{min} = 2 + 16 = 258$ Bytes.\\
$Size_{max} = Size_{min} + 65000 + ( 65000 \% 16 ) = 65260 $ Bytes.

\section{Protocollo}
L'autenticazione è stata eseguita sfruttando RSA in modalità OEAP.
\subsection{Server}
Il server ha una chiave privata e una pubblica (nota a priori per ogni client), è dotato di un database che contiene i nomi degli utenti, gli hash delle loro password e altre informazioni.\\
Il server conosce tutte le chiavi pubbliche degli utenti registrati al sistema.
\begin{itemize}
\item server.pem
\item server.pub
\item client\_*.pub
\item database's table user(name, hash, user\_type, reg\_date);
\end{itemize}

\subsection{Client}
Il client ha una chiave privata e pubblica e conosce la chiave pubblica del server.
\begin{itemize}
\item client\_\$\{username\}.pem
\item client\_\$\{username\}.pub
\item server.pub
\end{itemize}

\subsection{Autenticazione}
La procedura di autenticazione permette di stabilire una chiave di sessione tra client e server.
\begin{itemize}
\item \textbf{S} : Server
\item \textbf{C} : Client
\item $N_s$ : Server's nounce
\item $N_c$ : Client's nounce
\item $E_{s} ()$ : Encrypt with server's public key. 
\item $E_{c} ()$ : Encrypt with client's public key.
\item $K_c$ : Client's symmethric key.
\item $K_s$ : Server's symmethric key.
\item $K_{sc}$ : Combined session key.
\end{itemize}
Il protocollo di autenticazione usa la seguente sequenza di messaggi.
\begin{center}
    \begin{tabular}{ | c | c | p{6cm} |}
    \hline
    M1 & $S \rightarrow C$ : $N_s$ & Il server invia un nounce/challenge al client che si è appena connesso. \\ \hline
    M2 & $S \leftarrow C$ : $E_{s}$ ( $N_c$, $N_s$, User, PwdHash ) & Il client cripta con la chiave pubblica del server il nounce ricevuto, inserisce un proprio nounce, inserisce utente e hash della password. \\ \hline
    M3 & $S \rightarrow C$ : $E_{c}$ ( $N_c$, Response ) & Se l'utente non è già connesso, ha un nome valido e quindi è presente nel database e l'hash della password coincide con quello nel database, la risposta è positiva e viene autenticato, altrimenti viene rigettato. \\ \hline
    M4 & $S \leftarrow C$ : $E_{s}$ ( $K_c$ ) & Se il nounce viene verificato invia il seguente messaggio altrimenti la connessione viene chiusa.\\ \hline
    M5 & $S \rightarrow C$ : $E_{c}$ ( $K_s$ ) & Client e Server calcolano la chiave di sessione: $ K_{sc} = AES_{Na \oplus Nb}(K_s \oplus K_c) $ \\ \hline % non va bene da rivedere
    \end{tabular}
\end{center}
!! Manca la parte di key confirmation !! magari si spediscono gli xor dei nounce o qualcosa del genere\\

\subsection{Analisi Ban}

BAN logic statements:
\begin{center}
\begin{tabular}{|c | l|} \hline
$ S \mid \equiv X  $ & S believes X.\\
$ S \lhd X  $ & S sees X.\\
$ S \mid \sim X  $ & S once said X.\\
$ S \Rightarrow X $ & S controls X.\\ 
$ \#(X) $ & X is fresh.\\
$S {k \atop \leftrightarrow} C $ & K is a shared key between S and C.\\ 
$S {k \atop \leftrightharpoons} C $ & K is a shared secret between S and C.\\ 
${k  \atop \longmapsto} S $ & K is S's public key.\\
$\langle X \rangle_{Y}$ & X is combined with Y (secret).\\
$\{X\}_{K}$ & X has been cyphered with K.\\ \hline
\end{tabular}
\end{center}

Obiettivi:
\begin{itemize}
\item \framebox{$S \mid \equiv S {k \atop \leftrightarrow} C $} \framebox{$C \mid \equiv S {k \atop \leftrightarrow} C $} (\textit{key-authentication})
\item \framebox{$S \mid \equiv C \mid \equiv S {k \atop \leftrightarrow} C $} \framebox{$C \mid \equiv S \mid \equiv S {k \atop \leftrightarrow} C $} (\textit{key-confirmation})
\item \framebox{$ C \mid \equiv \#( S {k \atop \leftrightarrow} C) $} \framebox{$ S \mid \equiv \#( S {k \atop \leftrightarrow} C) $} (\textit{key-freshness})
\end{itemize}

Ipotesi:\\
\begin{itemize}
\item \textbf{(H1)}: $ S \mid \equiv \#(Na) $
\item \textbf{(H2)}: $ S \mid \equiv \#(Nb) $

\end{itemize}
Idealizzazione del protocollo:\\


\subsection{Aggiornamento chiave di sessione}
La chiave di sessione viene aggiornata ogni 15 minuti. La frequenza di aggiornamento è configurabile ma ha un minimo di 60 secondi. 
\end{document}
